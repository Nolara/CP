\section*{Aufgabe 1: }
\subsection*{Aufgabenteil 1)}
Die diskrete Fouriertransformation
\begin{equation}
  F_j=\sum_{l=0}^{N-1}\Omega_N^{j,l}f_l
\end{equation}
wird als schnelle Fouriertransformation mit $N=2^m$ Diskretisierungspunkten implementiert. Die Implementierung
des Algorithmus wird für
\begin{equation}
  f_l=\sqrt{1+l}
\end{equation}
und $m\in\{3,4\}$ verifiziert, indem die Fouriertransformation zudem direkt berechnet wird. Der Vergleich des Realteils und des Imaginärteils sind in Abbildung \ref{fig:plot1} dargestellt.
\begin{figure}[h]
\begin{subfigure}[c]{0.5\textwidth}
\includegraphics[width=\textwidth]{../../Blatt5/Plots/1_1_re_3.pdf}
\subcaption{Vergleich des Realteils für $m=3$}
\end{subfigure}
\begin{subfigure}[c]{0.5\textwidth}
\includegraphics[width=\textwidth]{../../Blatt5/Plots/1_1_im_3.pdf}
\subcaption{Vergleich des Imaginärteils für $m=3$}
\end{subfigure}
\begin{subfigure}[c]{0.5\textwidth}
\includegraphics[width=\textwidth]{../../Blatt5/Plots/1_1_re_4.pdf}
\subcaption{Vergleich des Realteils für $m=4$}
\end{subfigure}
\begin{subfigure}[c]{0.5\textwidth}
\includegraphics[width=\textwidth]{../../Blatt5/Plots/1_1_im_4.pdf}
\subcaption{Vergleich des Imaginärteils für $m=4$}
\end{subfigure}
\caption{Vergleich der Ergebnisse der FFT und der direkten Berrechnung}
\label{fig:plot1}
\end{figure}
Es lässt sich erkennen, dass die Ergebnisse beider Algorithmen übereinstimmen.

\subsection*{Aufgabenteil 2)}

Der Ergebnissvektor der FFT wird noch aufbereitet, indem die Elemete verschoben werden und anschließend
mit einem Phasenfaktor
\begin{equation}
  F'_j=\frac{\increment x}{2\pi}\text{exp}\left(-2\pi i\frac{x_{\text{min}}j}{L} \right) \text{d}x
\end{equation}
multipliziert werden. Mit dieser Implementierung der FFT wird die Fouriertransformierte
der Funktion
\begin{equation}
  f(x)=\text{exp}\left(-\frac{x^2}{2}\right)
\end{equation}
auf einem Intervall $x\in[-10,10]$ mit $m=7$ berechnet und mit der analytischen Lösung verglichen.
Letztere ergibt sich zu
\begin{align}
  F(k)&=\frac{1}{2\pi}\int_{-\infty}^{\infty}\text{exp}\left(-\frac{x^2}{2}\right)\text{exp}\left(
  ikx\right) \text{d}x \\
  &=\frac{1}{2\pi}\int_{-\infty}^{\infty}\text{exp}\left(-\frac{1}{2}(x^2-2ikx-k^2) -\frac{k^2}{2}\right) \text{d}x \\
  &=\frac{1}{2\pi}\text{exp}\left(-\frac{k^2}{2}\right)\int_{-\infty}^{\infty}\text{exp}\left(-\frac{1}{2}(x-ik)^2\right) \text{d}x \\
  &=\frac{1}{2\pi}\text{exp}\left(-\frac{k^2}{2}\right)\sqrt{2\pi}=\frac{1}{\sqrt{2\pi}}\text{exp}\left(-\frac{k^2}{2}\right) \:.
\end{align}
Der Vergleich der FFT zu dem analytischen Ergebniss ist in Abbildung \ref{fig:plot2} dargestellt.
\begin{figure}[h]
\begin{subfigure}[c]{0.5\textwidth}
\includegraphics[width=\textwidth]{../../Blatt5/Plots/1_2_Re.pdf}
\subcaption{Vergleich des Realteils.}
\end{subfigure}
\begin{subfigure}[c]{0.5\textwidth}
\includegraphics[width=\textwidth]{../../Blatt5/Plots/1_2_Im.pdf}
\subcaption{Vergleich des Imaginärteils.}
\end{subfigure}
\caption{Vergleich der Implementierung mit der analytischen Lösung.}
\label{fig:plot2}
\end{figure}
Es lässt sich erkennen, dass die Ergebnisse der FFT die analytische Lösungen gut approximieren und es bei dem Imaginärteil nur Abweichungen in der Größenordnung $10^{-14}$ gibt.


\subsection*{Aufgabenteil 3)}

Mit der FFT werden die komplexen Fourierkoeffizienten $c_n$ der $2\pi$-periodisch fortgesetzten Rechteckschwingung
\begin{equation}
  f(x)=
  \begin{cases}
    -1, & x\in[-\pi,0] \\
    1, & x\in(0,\pi)
  \end{cases}
\end{equation}
berechnet. Da die Funktion exakt periodisch ist, ergeben sich analytisch Delta Peaks der Höhe $\frac{2}{\pi}$ bei $-\pi$ und $\pi$. Der Vergleich dieser Funktion mit dem Betrag der durch die FFT berechneten Koeffizienten ist in Abbildung \ref{fig:plot3}
dargestellt.
\begin{figure}[H]
  \includegraphics[height=8cm]{../../Blatt5/Plots/1_3_Abs.pdf}
  \caption{Vergleich der Koeffizienten der FFT mit der analytischen Lösung.}
  \label{fig:plot3}
\end{figure}
Trotz einiger Abweichung aufgund der begrenzten Abtastrate lässt sich eine Tendenz hin zum Delta Peak erkennen.
