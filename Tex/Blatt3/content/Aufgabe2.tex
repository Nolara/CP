\section*{Aufgabe 2: Federkette }

\subsection*{Aufgabenteil a)}

  Um ein Programm zu schreiben, das für belibige N alle Eigenfrequenzen ausgibt,
  muss die zum Problem gehörige Matrix aufgestellt werden.
  Dazu werden die Bewegungegleichungen in eine Matrix geschrieben.
  \begin{equation}
    \vec{\ddot{x}} = M \vec{x}
  \end{equation}
  Beispielsweise für N=4 ist die erset Bewegungegleichung
  \begin{equation}
    \ddot{x}_0 = k_0 \frac{x_0}{m_0} - k_0 \frac{x_1}{m_1}
  \end{equation}
  Die Zweite, Dritte und Vierte lauten
  \begin{equation}
    m_1 \ddot{x}_1 = k_0 (\frac{x_1}{m_1}-\frac{x_0}{m_0}) + k_1 (\frac{x_2}{m_2}-\frac{x_1}{m_1} )
  \end{equation}
  \begin{equation}
    m_2 \ddot{x}_2 = + k_1 (\frac{x_2}{m_2}-\frac{x_1}{m_1} )+ k_2(\frac{x_3}{m_3}-\frac{x_2}{m_2})
  \end{equation}
  \begin{equation}
    m_3 \ddot{x}_3 = k_2 (\frac{x_3}{m_3}-\frac{x_2}{m_2}) + k_2 \frac{x_3}{m_3}
  \end{equation}
  Daraus lässt sich die Matrix $M$  zusammenfassen als
  \begin{equation}
    M =
    \left(\begin{array}{r r r r}
     k_0/m_0   & -k_0/m_1       & 0               &  0        \\
     -k_0/m_0  & (k_0+k_1)/m_1  &  -k_1/m_2       &  0        \\
      0        & -k_1/m_1       & (k_1+k_2)/m_2   & -k_2/m_3  \\
      0        & 0              &  -k_2/m_2       & k_2/m_3   \\
    \end{array}\right)
  \end{equation}
  Diese lässt sich nun auf belibige $N$ erweitern.



\subsection*{Aufgabenteil b)}

  Die Eigenfrequenzen $\omega_i$ des Systems sind mithilfe von Eigen/Eigenvalue
  und der Gleichung $\lambda = \omega^2$
  bestimmt worden
  $$ \omega_1 = 3.95439  $$
  $$ \omega_2 = 2.62228  $$
  $$ \omega_3 = 1.95382  $$
  $$ \omega_4 = 1.51083  $$
  $$ \omega_5 = 1.17586  $$
  $$ \omega_6 = 0.901489  $$
  $$ \omega_7 = 0.664894  $$
  $$ \omega_8 = 1.04981\cdot 10^{-8}  $$
  $$ \omega_9 = 0.453722  $$
  $$ \omega_{10} = 0.250241  $$
  Die größte Eigenfrequenz lieg wie im Hinweis angegeben bei
  $\omega=3.95439$.


