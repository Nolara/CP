\section*{Aufgabe 0: Verständnisfragen}

\textbf{Frage 1): Was unterscheidet eine echte Zufallszahl von einer Pseudozufallszahl? Wie können daraus
Zufallszahlen einer beliebigen Verteilung erzeugt werden?}\\
  Echte Zufallszahlen gibt es nur in Physikalischen Systemen, zum Beispiel bei einem
  quantenmechanischen Experiment. Da Computer deterministisch
  sind, können sie nur Pseudozufallszahlen mit Hilfe bestimmter Algorithmen erzeugen
  oder es müsste eine exakte Kopplung an ein physikalisches System geben, was meistens zu
  Aufwendig ist.

  Pseudozufallszahlen werden durch Pseudo-Zufallszahlengeneratoren (RPNGs)
  mit folgenden Eigenschaften erzeugt:
  \begin{itemize}
  \item[1)] Erzeugung gleichverteilter Zufallszahlen
  \item[2)] iterativ, benötigen Startwert oder Seed
  \item[3)] Bei gleichem Seed werden reproduzierbare Sequenzen erstellt
  \item[4)] nach Periodenlänge wiederholen sich RPNGs (daher Zufallszahl nicht "echt")
  \end{itemize}

  Aus diesen Zufallszahlen lassen sich Zufallszahlen einer beliebigen
  Verteilung erstellen. Dazu gibt es mehrere Methoden, wie zum Beispiel:
  \begin{itemize}
  \item[1)] Transformations-Methode

      Mit gleichverteilten Zufallszahlen $x \in{x_1,x_2}$ (nach Verteilung $p(x)$) und
      gegebener monotoner Funktion $y(x)$ ergeben sich Zahlen $y$ nach Verteilung
      $\tilde{p}(y)$ mit
      \begin{equation*}
        \tilde{p}(y) = p(x) |\frac{dx}{dy}|
      \end{equation*}

  \item[2)] Box-Müller-Verfahren

      Gaußverteilung ($\tilde{p}(y)=\frac{1}{\sqrt{2 \pi}} \text{exp}(-y^2/2)$) kann nicht mit 1) konstruiert werden.
      $\rightarrow$ Nutze 2D  Verallgemeinerung, für die gilt:
      \begin{equation*}
        \left| \frac{\partial(x_1,x_2)}{\partial(y_1,y_2)} \right| = \left(\frac{1}{\sqrt{2\pi}}e^{-y_1^2/2}\right)\left(\frac{1}{\sqrt{2\pi}}e^{-y_2^2/2}\right)
      \end{equation*}
      $\rightarrow$ es werden zwei gaußverteilte Zahlen konstruiert:
      \begin{equation*}
        y_1 = \sqrt{-2 \log{x_1}} \cos{(2 \pi x_2)}
      \end{equation*}
      \begin{equation*}
        y_2 = \sqrt{-2 \log{x_1} } \sin{(2 \pi x_2)}
      \end{equation*}
  \end{itemize}

\newpage
\textbf{Frage 2): Was sind die grundlegenden Eigenschaften eines Markov-Prozesses?}\\

  \begin{itemize}
    \item Zustände können diskret ($i$) oder kontinuierlich ($\vec{r}$) sein
    \item stochastische Übergänge zwischen den Zuständen mit gegebener Wahrscheinlichkeit und kontinuierliche Zeit t
    \item Übergangswahrscheinlichkeit nur von Anfangszustand und Endzustand abhängig
    \item Ergebnis des Prozesses ist Aufenthaltswahrscheinlichkeit $p_i(t)$ bzw $p(\vec{r},t)dV$
  \end{itemize}

\textbf{Frage 3): Erklären Sie den Begriff detailed balance bzw. detailliertes Gleichgewicht.
Ist die Übergangsmatrix durch diese Bedingung eindeutig definiert?}\\

  Gesucht sind stationäre Zustände $\vec{p}_{s}$ der Master Gleichung
  $\vec{p}^T_{k+1} = \vec{p}^T_k \underline{\underline{M}}$ für $k \rightarrow \infty$
  \begin{equation*}
    \vec{p}^T_{s} = \vec{p}^T_s \underline{\underline{M}}
  \end{equation*}
  Dabei muss $\vec{p}^T_{s}$ Links-Eigenvektor von $\underline{\underline{M}}$
  zum EW 1 sein. Damit alles unabhängig von der Wahl von $\vec{p}^T_0$ ist, muss
  1 der größte EW sein und der EV darf nicht entartet sein.

  $\Rightarrow \underline{\underline{M}}$ muss richtig gewählt werden, damit die Gleichung erfüllt
  werden kann.
  Eine hinreichende Bedingung ist dabei die detailed balance (DB):
  \begin{equation*}
    j^S_{i,j}=0\,\, \forall i,j \,\,\,\,\Leftrightarrow \,\,\,\, p_s^i M_{ij} = p_s^j M_{ji} \,\,\forall i,j
  \end{equation*}
  Hier wird angenommen, dass es keine Ströme im System gibt, also dass ein
  thermodynamisches Gleichgewicht besteht.

  Die Übergangsmatrix $\underline{\underline{M}}$ ist durch diese Bedingung nicht
  eindeutig definiert.
