\section*{Aufgabe 1: Monte-Carlo-Simulation eines einzelnen Spins (Abgabe bis zum 3.07.2020)}

In dieser Aufgabe wird mit Hilfe des Metropolis-Algorithmus ein einzelner Spin $\sigma=\pm1$ mit der Energie
\begin{equation*}
  \symcal{H}=-\sigma H
\end{equation*}
im äußeren Magnetfeld $H$ simuliert. Dies ist in der Aufgabe1.cpp implementiert, dabei wird $k_{\symup{B}} T=1$ gesetzt.
Es werden für die numerischen Berechnungen Monte-Carlo-Simulationen für $10^4$ Werte für $H\in[-5,5]$ mit je $N=10^5$ Schritten durchgeführt.
Die numerischen Ergebnisse werden mit dem analytischen Ergebnis
\begin{equation*}
  m=\tanh\left(\beta H \right)
\end{equation*}
in der Abbildung \ref{Magnetisierung} graphisch verglichen.
Der Vergleich der numerischen und analytischen Ergebnisse zeigt, dass der Metropolis-Algorithmus das analytische Ergebnis genau approximiert.

\begin{figure}[H]
\includegraphics[width=0.8\textwidth]{../../Blatt9/Plots/plot_1.pdf}
\centering
\caption{Graphische Darstellung und Vergleich der analytischen und numerischen Berechnung der Magnetisierung $m$ in Abhängigkeit des äußeren Magentfeldes $H$.}
\label{Magnetisierung}
\end{figure}
