\documentclass[
  bibliography=totoc,     % Literatur im Inhaltsverzeichnis
  captions=tableheading,  % Tabellenüberschriften
  titlepage=firstiscover, % Titelseite ist Deckblatt
]{scrartcl}
\newcommand{\pvec}[1]{\vec{#1}\mkern2mu\vphantom{#1}}
% Paket float verbessern
\usepackage{scrhack}

% Warnung, falls nochmal kompiliert werden muss
\usepackage[aux]{rerunfilecheck}

% unverzichtbare Mathe-Befehle
\usepackage{amsmath}
% viele Mathe-Symbole
\usepackage{amssymb}
% Erweiterungen für amsmath
\usepackage{mathtools}

% Fonteinstellungen
\usepackage{fontspec}
% Latin Modern Fonts werden automatisch geladen
% Alternativ zum Beispiel:
%\setromanfont{Libertinus Serif}
%\setsansfont{Libertinus Sans}
%\setmonofont{Libertinus Mono}

% Wenn man andere Schriftarten gesetzt hat,
% sollte man das Seiten-Layout neu berechnen lassen
\recalctypearea{}

% deutsche Spracheinstellungen
\usepackage{polyglossia}
\setmainlanguage{german}


\usepackage[
  math-style=ISO,    % ┐
  bold-style=ISO,    % │
  sans-style=italic, % │ ISO-Standard folgen
  nabla=upright,     % │
  partial=upright,   % ┘
  warnings-off={           % ┐
    mathtools-colon,       % │ unnötige Warnungen ausschalten
    mathtools-overbracket, % │
  },                       % ┘
]{unicode-math}

% traditionelle Fonts für Mathematik
\setmathfont{Latin Modern Math}
% Alternativ zum Beispiel:
%\setmathfont{Libertinus Math}

\setmathfont{XITS Math}[range={scr, bfscr}]
\setmathfont{XITS Math}[range={cal, bfcal}, StylisticSet=1]

% Zahlen und Einheiten
\usepackage[
  locale=EN,                   % deutsche Einstellungen
  separate-uncertainty=true,   % immer Fehler mit \pm
  per-mode=symbol-or-fraction, % / in inline math, fraction in display math
]{siunitx}

% chemische Formeln
\usepackage[
  version=4,
  math-greek=default, % ┐ mit unicode-math zusammenarbeiten
  text-greek=default, % ┘
]{mhchem}

% richtige Anführungszeichen
\usepackage[autostyle]{csquotes}

% schöne Brüche im Text
\usepackage{xfrac}

% Standardplatzierung für Floats einstellen
\usepackage{float}
\floatplacement{figure}{htbp}
\floatplacement{table}{htbp}

% Floats innerhalb einer Section halten
\usepackage[
  section, % Floats innerhalb der Section halten
  below,   % unterhalb der Section aber auf der selben Seite ist ok
]{placeins}

% Seite drehen für breite Tabellen: landscape Umgebung
\usepackage{pdflscape}

% Captions schöner machen.
\usepackage[
  labelfont=bf,        % Tabelle x: Abbildung y: ist jetzt fett
  font=small,          % Schrift etwas kleiner als Dokument
  width=0.9\textwidth, % maximale Breite einer Caption schmaler
]{caption}
% subfigure, subtable, subref
\usepackage{subcaption}

% Grafiken können eingebunden werden
\usepackage{graphicx}
% größere Variation von Dateinamen möglich
\usepackage{grffile}

% schöne Tabellen
\usepackage{booktabs}

% Verbesserungen am Schriftbild
\usepackage{microtype}

% Literaturverzeichnis
\usepackage[
  backend=biber,
]{biblatex}
% Quellendatenbank
\addbibresource{lit.bib}
\addbibresource{programme.bib}

% Hyperlinks im Dokument
\usepackage[
  unicode,        % Unicode in PDF-Attributen erlauben
  pdfusetitle,    % Titel, Autoren und Datum als PDF-Attribute
  pdfcreator={},  % ┐ PDF-Attribute säubern
  pdfproducer={}, % ┘
]{hyperref}
% erweiterte Bookmarks im PDF
\usepackage{bookmark}

% Trennung von Wörtern mit Strichen
\usepackage[shortcuts]{extdash}
%Grafiken an position H
\usepackage{float}
%immer noindent
\setlength\parindent{0pt}

\author{%
  Sinja Behrensmeier\\%
  \href{mailto:Sinja.Behrensmeier@tu-dortmund.de}{Sinja.Behrensmeier@tu-dortmund.de}%
  \texorpdfstring{\and}{,}%
  Yvonne Ribbeheger\\%
  \href{mailto:Yvonne.Ribbeheger@tu-dortmund.de}{Yvonne.Ribbeheger@tu-dortmund.de}%
  \texorpdfstring{\and}{,}%
  Lara Nollen\\%
  \href{mailto:Lara.Nollen@tu-dortmund.de}{Lara.Nollen@tu-dortmund.de}%
}
\publishers{TU Dortmund – Fakultät Physik}


\subject{Computational Physics}
\title{Übungsblatt 1}
\date{%
  Abgabe: 1.05.2020
}

\begin{document}

\maketitle
\thispagestyle{empty}
%\tableofcontents
\newpage

\section*{Aufgabe 0: Verständnisfragen}

\textbf{Frage 1): Welche Möglichkeiten gibt es lineare Gleichungssysteme (mit eindeutiger Lösung) zu lösen?}\\

Es gibt mehrere Möglichkeiten:
\begin{itemize}
  \item Berechnung der Inversen der Matrix $\symbf{A}$ und Bestimmung des Lösungsvektors über $\vec{x}=\symbf{A}^{-1}\vec{b}$
  \item Gaußsches Verfahren
  \item LU Zerlegung \to sehr effizient und stabil: $\symbf{A}=\symbf{P}\cdot \symbf{L}\cdot \symbf{U}$ mit
  der orthogonalen Pivotisierungsmatrix $\symbf{P}$, der unteren Dreiecksmatrix $\symbf{L}$ und der
  oberen Dreiecksmatrix $\symbf{U}$
  \item Mit der Singulärwertzerlegung
\end{itemize}

\textbf{Frage 2): Wieso ist im Allgemeinen eine Pivotierung notwendig?}\\
Da es im Allgemeinen möglich ist, dass es auf der Diagonalen eine 0 gibt, so dass nicht durch das Element $A_{i,i}=0$ dividiert werden kann. Somit kann es nötig sein, zunächst die Zeilen geeignet zu vertauschen.


\section*{Aufgabe 1: Basiswechsel mit LU-Zerlegung}
\subsection*{Aufgabenteil a)}
Es handelt sich um ein hcp Gitter (Hexagonal dichteste Kugelpackung,"hexagonal close packed").
\subsection*{Aufgabenteil b)}
Das Problem lässt sich als lineares Gleichungssystem darstellen, indem die drei primitiven Gittervektoren zu
der Matrix
\begin{equation}
  \symbf{A}=(\vec{a_1},\vec{a_2},\vec{a_3})=
  \begin{pmatrix}
    \frac{1}{2} & -\frac{1}{2} & 0 \\
    \frac{\sqrt{3}}{2} & \frac{\sqrt{3}}{2} & 0 \\
    0 & 0 & 1 \\
  \end{pmatrix} \\
\end{equation}
zusammengefasst werden, sodass sich das Problem durch die Gleichung
\begin{equation}
  \symbf{A}\pvec{x}'=\vec{x}
  \label{eqn:gitter}
\end{equation}
beschreiben lässt.
Dabei ist der Vektor $\vec{x}$ der Fehlstelle im Kristall durch
\begin{equation*}
  \pvec{x}=
  \begin{pmatrix}
    2 \\
    0 \\
    2 \\
  \end{pmatrix} \\
\end{equation*}
gegeben.
Durch die Verwendung der LU Zerlegung Eigen::PartialPivLu aus der Bibliothek Eigen lässt sich das Gleichungssystem lösen, wobei sich der Lösungsvektor
\begin{equation*}
  \pvec{x}'=
  \begin{pmatrix}
    2 \\
    -2 \\
    2 \\
  \end{pmatrix} \\
\end{equation*}
ergibt. Desweiteren ergeben sich die Matrizen
\begin{equation*}
  \symbf{P}=
  \begin{pmatrix}
    0 & 1 & 0 \\
    1 & 0 & 0 \\
    0 & 0 & 1 \\
  \end{pmatrix} \: ,
  \symbf{L}=
  \begin{pmatrix}
    1 & 0 & 0 \\
    0.57735 & 1 & 0 \\
    0 & 0 & 1 \\
  \end{pmatrix}
  \text{und }
  \symbf{U}=
  \begin{pmatrix}
    0.866025 & 0.866025 & 0 \\
    0 & -1 & 0 \\
    0 & 0 & 1 \\
  \end{pmatrix} \: .
\end{equation*}

\subsection*{Aufgabenteil c)}
Zur erneuten Bestimmung der Koordinaten der Störstelle $\pvec{y}'$ lassen sich die bereits
in Aufgabenteil b) berechneten Matrizen $\symbf{P}$, $\symbf{L}$ und $\symbf{U}$ wiederverwenden, für deren Berechnung ein Aufwand von $\symcal{O}(n^3)$ nötig ist für eine Matrix der Größe $n$x$n$.
Durch Vorwärtseinsetzen mit
\begin{equation}
  z_1=b_1 \: , \quad z_i=b_i-\sum_{j=1}^{i-1}l_{ij}z_j\;\quad i \in [2,n]
\end{equation}
und Rückwertseinsetzen durch
\begin{equation}
  x_n=\frac{z_n}{u_{n,n}} \: , \quad x_i=(z_i-\sum_{j=1+i}^{n}u_{ij}x_j)/u_{i,i}\;\quad i \in [1,n-1]
\end{equation}
mit einem Aufwand von $\symcal{O}(n^2)$ %Kenne den Aufwand nicht
lässt sich damit die Störstelle bestimmen.
Die Koordinaten von $\vec{y}'$ werden somit zu
\begin{equation*}
  \pvec{x}'=
  \begin{pmatrix}
    3 \\
    1 \\
    3 \\
  \end{pmatrix} \\
\end{equation*}
errechnet.


\subsection*{Aufgabenteil d)}
Mit Hilfe von Eigen::PartialPivLu wird nun eine LU-Zerlegung für die Basis $\{\vec{a_3},\vec{a_2},\vec{a_1}\}$ durchgeführt. In dem Vergleich der beiden Aufgabenteile a) und d) lässt sich festellen, dass der Lösungsvekor gleich ist, was zu erwarten ist.
Es ergeben sich in Aufgabenteil d) folgende Matrizen durch die LU-Zerlegung
\begin{equation*}
  \symbf{P'}=
  \begin{pmatrix}
    0 & 0 & 1 \\
    0 & 1 & 0 \\
    1 & 0 & 0 \\
  \end{pmatrix} \: ,
  \symbf{L'}=
  \begin{pmatrix}
    1 & 0 & 0 \\
    0 & 1 & 0 \\
    0 & -0.57735 & 1 \\
  \end{pmatrix}
  \text{und }
  \symbf{U'}=
  \begin{pmatrix}
    1 & 0 & 0 \\
    0 & 0.866025 & 0.866025 \\
    0 & 0 & 1 \\
  \end{pmatrix} \: .
\end{equation*}
Es fällt auf, dass bei dem Wechsel der Basis von $\{\vec{a_1},\vec{a_2},\vec{a_3}\}$ in die Basis $\{\vec{a_3},\vec{a_2},\vec{a_1}\}$ ebenfalls die Zeilen der Permutationsmatrix und die Einträge in den beiden Dreiecksmatrizen wechseln.

\subsection*{Aufgabenteil e)}

Wenn alle primitiven Gittervektoren paarweise orthogonal sind, dann handelt es sich bei der Matrix $\symbf{A}$  um eine orthogonale Matrix. Bei dieser Art von Matrizen ist die Inverse einfach durch die Transponierte gegeben, also
\begin{equation}
  \symbf{A}^{-1}=\symbf{A}^T \: ,
\end{equation}
sodass sich das Problem aus Gleichung \eqref{eqn:gitter} durch Matrixmultiplikation in der Form
\begin{equation}
  \pvec{x}'=\symbf{A}^T\vec{x}
\end{equation}
lösen lässt mit einem Aufwand von der Ordnung $\symcal{O}(n^2)$.



\section*{Aufgabe 2: Ausgleichsrechnung}

Gegeben seien folgende ($x,y$)-Datenpunkte:
\begin{table}[H]
\centering
\label{tab:Datenpunkte}
\begin{tabular}{c c c c c c c c c c c}
\toprule
x & 0 & 2.5 & -6.3 & 4 & -3.2 & 5.3 & 10.1 & 9.5 & -5.4 & 12.7 \\
\midrule
y & 4 & 4.3 & -3.9 & 6.5 & 0.7 & 8.6 & 13 & 9.9 & -3.6 & 15.1 \\
\bottomrule
\end{tabular}
\end{table}
Berechnen Sie eine Ausgleichsgerade
\begin{align}
  y(x) &=mx+n,
  \label{eqn:Ausgleichsgerade}
  \intertext{die den quadratischen Fehler}
  R &= \sum_{i}(mx_i +n-y_i)^2
\end{align}
für die gegebenen Datenpunkte durch das Lösen eines Gleichungssystem minimiert.

\subsection*{Aufgabenteil a)}
Das Problem lässt sich in Form eines überbestimmten Gleichungssystems des Form
\begin{equation}
  \symbf{A}\vec{x}=\vec{b}
  \label{eqn:Lgs}
\end{equation}
beschreiben, wobei die Matrix $\symbf{A}$ aus dem Zeilenvektor der Datenpunkte $x$ aus der Aufgabenstellung sowie einem
Hilfsvektor mit je einer 1 als Eintrag besteht, $\vec{x}$ aus dem Wert der Steigung m und dem y-Achsenabschnitt
n besteht und $\vec{b}$ der Zeilenvektor der Datenpunkte $y$ aus der Aufgabenstellung ist, also
\begin{equation}
  \begin{pmatrix}
    0 & 1 \\
    2.5 & 1 \\
    -6.3 & 1 \\
    4 & 1 \\
    -3.2 & 1 \\
    5.3 & 1 \\
    10.1 & 1 \\
    9.5 & 1 \\
    -5.4 & 1 \\
    12.7 & 1 \\
  \end{pmatrix}
  \cdot
  \begin{pmatrix}
    m \\
    n \\
  \end{pmatrix}
  =
  \begin{pmatrix}
    4 \\
    4.3 \\
    -3.9 \\
    6.5 \\
    0.7 \\
    8.6 \\
    13 \\
    9.9 \\
    -3.6 \\
    15.1 \\
  \end{pmatrix} \: .
\end{equation}

\subsection*{Aufgabenteil b)}
Durch Multiplikation der Gleichung \eqref{eqn:Lgs} mit $\symbf{A}^{T}$ von links auf beiden Seiten, lässt sich das überbestimmte System in die Form
\begin{equation}
  \symbf{A}^{T}\symbf{A}\vec{x}=\symbf{A}^{T}\vec{b}
  \label{eqn:lgs2}
\end{equation}
überführen mit der quadratischen Matrix $\symbf{A}^{T}\symbf{A}$, sodass die Gleichung in diesem Beispiel konkret die Form
\begin{equation}
  \begin{pmatrix}
    482.98 & 29.2 \\
    29.2 & 10 \\
  \end{pmatrix}
  \cdot
  \begin{pmatrix}
    m \\
    n \\
  \end{pmatrix}
  =
  \begin{pmatrix}
    541.22 \\
    54.6 \\
  \end{pmatrix} \: .
\end{equation}

\subsection*{Aufgabenteil c)}
Durch die Verwendung der LU Zerlegung Eigen::PartialPivLu aus der Bibliothek Eigen lässt sich das Gleichungssystem lösen, wobei sich der Lösungsvektor
\begin{equation*}
  \vec{x}=
  \begin{pmatrix}
    0.959951\\
    2.65694\\
  \end{pmatrix} \\
\end{equation*}
ergibt.
\subsection*{Aufgabenteil d)}
Die Datenpunkte aus der Aufgabenstellung sind zusammen mit der errechneten Ausgleichsgerade in Abbildung \ref{fig:gerade} dargestellt.
\begin{figure}[H]
  \centering
  \includegraphics[height=8cm]{../Blatt1/Plots/2.pdf}
  \caption{Datenpunkte $x$ und $y$ sowie die errechnete Ausgleichsgerade.}
  \label{fig:gerade}
\end{figure}
Es lässt sich erkennen, dass die Gerade tatsächlich gut zu den Daten passt.

\end{document}
