\section*{Aufgabe 2: BFGS-Verfahren }


  \begin{figure}
    \centering
    \includegraphics[width=10cm]{../../Blatt6/Plots/plot_A2_a}
    \caption{Die berechneten f($x_k$) in Abhängigkeit der Iterationsanzahl k für die  exakte inverse Hesse-Matrix.}
    \label{fig:A2a1}
  \end{figure}
  \begin{figure}
    \centering
    \includegraphics[width=10cm]{../../Blatt6/Plots/plot_A2_a1}
    \caption{Die berechneten f($x_k$), die kleiner als 6 sind, in Abhängigkeit der Iterationsanzahl k für die  exakte inverse Hesse-Matrix.}
    \label{fig:A2a2}
  \end{figure}
  \begin{figure}
    \centering
    \includegraphics[width=10cm]{../../Blatt6/Plots/plot_A2_b}
    \caption{Die berechneten f($x_k$) in Abhängigkeit der Iterationsanzahl k für die Diagonalmatrix auf deren Diagonale die inversen Diagonalelemente der Hesse-Matrix stehen.}
    \label{fig:A2b1}
  \end{figure}
  \begin{figure}
    \centering
    \includegraphics[width=10cm]{../../Blatt6/Plots/plot_A2_b1}
    \caption{Die berechneten f($x_k$), die kleiner als 6 sind, in Abhängigkeit der Iterationsanzahl k für die Diagonalmatrix, auf deren Diagonale die inversen Diagonalelemente der Hesse-Matrix stehen.}
    \label{fig:A2b2}
  \end{figure}
  \begin{figure}
    \centering
    \includegraphics[width=10cm]{../../Blatt6/Plots/plot_A2_c}
    \caption{Die berechneten f($x_k$) in Abhängigkeit der Iterationsanzahl k für ein Vielfaches der Einheitsmatrix mit einem Vorfaktor f($\vec{x}_0$).}
    \label{fig:A2c1}
  \end{figure}
  \begin{figure}
    \centering
    \includegraphics[width=10cm]{../../Blatt6/Plots/plot_A2_c1}
    \caption{Die berechneten f($x_k$), die kleiner als 6 sind, in Abhängigkeit der Iterationsanzahl k für ein Vielfaches der Einheitsmatrix mit einem Vorfaktor f($\vec{x}_0$)}
    \label{fig:A2c2}
  \end{figure}

  In den Abbildungen \ref{fig:A2a1}, \ref{fig:A2b1} und \ref{fig:A2c1} sind die berechneten Werte für $f(\vec{x}_k)$ gegen $k$ aufgetragen. Für einen
  besseren Vergleich wurde in den Abbildungen \ref{fig:A2a2}, \ref{fig:A2b2} und \ref{fig:A2c2} die y-Achse auf Werte unter 6 eingeschränkt. Die blaue
  Linie zeichnet dabei das Minimum der Funktion $f(\vec{x}_\text{min})=0$ für $\vec{x}_\text{min}=(1,1)^T$ ein, welchens analytisch bestimmt wurde.

  Die Endergebnisse der numerisch bestimmten Werte sind für die  exakte inverse Hesse-Matrix
  $$k_a = 77 $$
  $$ \vec{x}_\text{min,a}= (0.999992 , 0.999992))^T $$
  $$f(\vec{x}_\text{min,a})=6.86165e-11, $$
  für die Diagonalmatrix, auf deren Diagonale die inversen Diagonalelemente der Hesse-Matrix stehen,
  $$k_b= 123$$
  $$\vec{x}_\text{min,b}=(1.00001 , 1.00001)^T$$
  $$f(\vec{x}_\text{min,b})=9.59826e-10$$
  und für das Vielfaches der Einheitsmatrix mit einem Vorfaktor f($\vec{x}_0$)
  $$k_c= 120$$
  $$\vec{x}_\text{min,c}=(0.999964 , 0.999964)^T$$
  $$ f(\vec{x}_\text{min,c})=1.85273e-09.$$

  Die Iterationsanzahl für die erste Mäglichkeit ist geringer als die für die Dritte, die wiederrum geringer ist als die für die Zweite.
  Das Ergebniss mit der ersten Matrix ist genauer als das mit der zweiten Matix, welches wiederum genauer ist als das mit der dritten Methode.
