\section*{Aufgabe 1: Runge-Kutta-Verfahren}

  \subsection*{a)}

    \begin{figure}
      \centering
      \includegraphics[width=10cm]{./Plots/plot_A1_1}
      \caption{Lösung der Newtonsche Bewegungsgleichung für die AB \ref{eqn:AB1} und $h=1/100$.}
      \label{fig:A11}
    \end{figure}
    \begin{figure}
      \centering
      \includegraphics[width=10cm]{./Plots/plot_A1_1b}
      \caption{Lösung der Newtonsche Bewegungsgleichung für die AB \ref{eqn:AB1} und $h=1/10$.}
      \label{fig:A11b}
    \end{figure}
    \begin{figure}
      \centering
      \includegraphics[width=10cm]{./Plots/plot_A1_1c}
      \caption{Lösung der Newtonsche Bewegungsgleichung für die AB \ref{eqn:AB1} und $h=1/1000$.}
      \label{fig:A11c}
    \end{figure}

    \begin{figure}
      \centering
      \includegraphics[width=10cm]{./Plots/plot_A1_2}
      \caption{Lösung der Newtonsche Bewegungsgleichung für die AB \ref{eqn:AB2} und $h=1/100$.}
      \label{fig:A12}
    \end{figure}
    \begin{figure}
      \centering
      \includegraphics[width=10cm]{./Plots/plot_A1_3}
      \caption{Lösung der Newtonsche Bewegungsgleichung für die AB \ref{eqn:AB3} und $h=1/100$.}
      \label{fig:A13}
    \end{figure}

    In den Abbildungen \ref{fig:A11}, \ref{fig:A11b}, \ref{fig:A11c}, \ref{fig:A12} und \ref{fig:A13} sind Lösungen der
    Newtonsche Bewegungsgleichung für ein Teilchen in dem Kraftfeld $\vec{F}(\vec{r})$ graphisch
    dargestellt.
    \begin{equation}
      \dot{\vec{r}} = \vec{v}
    \end{equation}
    \begin{equation}
      \dot{\vec{v}}= \vec{F}(\vec{r})/m
    \end{equation}
    \begin{equation}
      \vec{F}(\vec{r}) = - \vec{r} m \omega^2
    \end{equation}
    Dabei wurde $\omega=1$ gesegtzt.
    Es wurden drei unterschiedliche Anfangsbedingungen gewählt
    \begin{equation}
      \vec{r}_0 \text{ belibig},\,\,\,\, \vec{v}_0=0,
      \label{eqn:AB1}
    \end{equation}
    \begin{equation}
      \vec{r}_0\text{ belibig},\,\,\,\,  \vec{v}_0= \vec{r}_0,
      \label{eqn:AB2}
    \end{equation}
    und
    \begin{equation}
      \vec{r}_0,\, \vec{v}_0 \text{ belibig}
      \label{eqn:AB3}
    \end{equation}

    In den Abbildungen \ref{fig:A11}, \ref{fig:A11b} und \ref{fig:A11c} wurden
    die Anfangsbedingungen aus Gleichung \ref{eqn:AB1} benutzt und die Schrittweite $h$ variiert.
    Zu erkennen ist, dass die Amplituden mit der Zeit größer werden und dass dieser
    Anstieg mit kleinerem $h$ kleiner wird.

    In den Abbildungen \ref{fig:A11},  \ref{fig:A12} und \ref{fig:A13} wurde
    für die selbe Schrittweite $h=1/100$ die Anfangsbedingungen aus
    den Gleichungen \ref{eqn:AB1}, \ref{eqn:AB2} und \ref{eqn:AB3} genutzt.
    Dabei fällt auf, dass für die Anfangsbedingung aus \ref{eqn:AB3} im Gegensatz
    zu den anderen beiden die einzelnen $r_i$ und $v_i$ Komponenten zuweinander
    verschoben schwingen.

    In der Abbildung \ref{fig:A11} ist eine Frequenz von $T= 2\pi$ zu erkennen.


  \subsection*{b)}

    \begin{figure}
      \centering
      \includegraphics[width=10cm]{./Plots/plot_A1_abw}
      \caption{Abweichung $|\vec{r}_0 - \vec{r}_i|$ gegen die Schrittweite $h$ nach 10 Schwingungen.}
      \label{fig:A1abw}
    \end{figure}
    Da wie in der a) bestimmt die Frequenz $2\pi$ ist, ist die Zeit, die nach
    10 Schwinungen vergangen ist, $t_\text{stopp}=10\cdot2\pi$. In der Abbildung \ref{fig:A1abw}
    ist die Abweichung $|\vec{r}_0 - \vec{r}(t_\text{stopp})|$ gegen die Schrittweite $h$
    aufgetragen.

  \subsection*{c)}

    \begin{figure}
      \centering
      \includegraphics[width=10cm]{./Plots/plot_A1_E}
      \caption{Gesammtenergie für AB \ref{eqn:AB1} und verschiedene Schrittweiten.}
      \label{fig:A1E}
    \end{figure}
    \begin{figure}
      \centering
      \includegraphics[width=10cm]{./Plots/plot_A1_E2}
      \caption{Gesammtenergie für verschiedene Anfangsbedingungen und Schrittweite $h=1/100$.}
      \label{fig:A1E2}
    \end{figure}

    In der Abbildunge \ref{fig:A1E} ist die Gesammtenergie für die
    Anfangsbedingungen \ref{eqn:AB1} für unterschiedliche Schrittweiten
    gegen die Zeit dargestellt. Die Energie nimmt mit der Zeit bei allen drei
    Schrittweiten zu, je kleiner die Schrittweite, desto geringer ist der Energieanstieg
    und desto genauer ist die Lösung.

    In der Abbildunge \ref{fig:A1E2} ist die Gesammtenergie für die drei verschiedenen
    Anfangsbedingungen gegen die Zeit aufgetragen. Die Schrittweite ist hier $h=1/100$.

    Die Gesammtenergie berechnet sich nach
    \begin{equation}
      E_\text{ges}(t) = \frac{m}{2}(v^2(t)+r^2(t)).
    \end{equation}
