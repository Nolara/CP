\section*{Aufgabe 2: }

\subsection*{Aufgabenteil a)}
Zur Berechnung des elektrostatische Potentials auf der x-Achse, wird das Integral zunächst einheitenlos
gemacht indem alle Längen in Einheiten von a angegeben werden und das Potential in Einheiten
von $\frac{\rho_0}{4\pi\epsilon_0}$, also
\begin{align}
  \{x,x',y',z'\}&\mapsto \Big\{\frac{x}{a},\frac{x'}{a},\frac{y'}{a},\frac{z'}{a}\Big\} \\
  \Phi &\mapsto \frac{\Phi}{\sfrac{\rho_0}{4\pi\epsilon_0}} \: ,
\end{align}
sodass sich das Integral zur Bestimmung des elektrostatischen Potentials in entsprechenden Einheiten für die Ladungsverteilung
\begin{equation}
  \rho(x,y,z)=
  \begin{cases}
    \rho_0, & \lvert x \rvert < a, \lvert y \rvert < a, \lvert z \rvert < a \\
    0, & \text{sonst}
  \end{cases}
\end{equation}
zu
\begin{equation}
  \Phi(x)=\int_{-1}^{1}\text{d}x'\int_{-1}^{1}\text{d}y'\int_{-1}^{1}\text{d}z'\frac{1}{\sqrt{(x-x')^2+y'^2+z'^2}}
\end{equation}
ergibt.
Zur Berechnung dieses Integrals wird der Würfel in $n^3$ kleine Würfel mit Kantenlängen
$h=\frac{2}{n}$ unterteilt, bei denen jeweils der Funktionswert in der Mitte berechnet wird. Also wird einem Würfel im Bereich $[x'_a,x'_a]\times[y'_a,y'_b] \times[z'_a,z'_b]$ der Funktionswert $f\left(x,\frac{x'_a+x'_b}{2},\frac{y'_a+y'_b}{2},\frac{z'_a+z'_b}{2}\right)$ zugewiesen, wobei die Funktion $f(x,x',y',z')$
dem Integranten entspricht. Da nie ein Funktionswert am "Rand" eines Würfels berechnet wird, sind zumindest
$y'^2$ und $z'^2$ stets ungleich 0 wenn man die Würfelkantenb in eine gerade Anzahl von Intervallen teilt, da die Achse mit $\{y',z'\}=\{0,0\}$ stets auf dem Rand der kleinen Würfel liegt, sodass es zu keinen Problemen durch eventuelle Singularitäten bei $x'=x$ kommt. \\
Mit dieser Methode wird mit $n=1000$ das Integral zunächst außerhalb des Würfels für die $x$-Werte
$x$=0.1$m$ mit $m\in\{11,12,...,80\}$ und anschließend innerhalb des Würfels mit $m\in{0,1,..,10}$ berechnet.\\ Zudem wird die Asymptotik abgeschätzt, indem eine Multipolentwicklung bis zur ersten nichtverschwindenden Ordnung durchgeführt wird, also in der Form
\begin{equation}
  \frac{1}{\sqrt{(x-x')^2+y'^2+z'^2}}\approx \frac{1}{\lvert x \rvert}+ x'\frac{x}{\lvert x \rvert^3} \: ,
\end{equation}
sodass sich das Integral in dieser Näherung für $x>>x'$ durch
\begin{equation}
  \Phi(x)\approx\int_{-1}^{1}\text{d}x'\int_{-1}^{1}\text{d}y'\int_{-1}^{1}\text{d}z'\frac{1}{\lvert x \rvert}+x'\frac{x}{\lvert x \rvert^3} = \frac{8}{\lvert x \rvert}
\end{equation}
analytisch berechnen lässt, wobei der Term linear in $x'$, also das Dipolmoment, wegfällt, da es eine ungerade Funktion ist die über ein gerades Intervall integriert wird. Somit bleibt in diesem Fall in erster Ordnung der Monopolterm $\frac{8}{\lvert x \rvert}$ übrig, der zusammen mit dem numerische errechneten Werten in Abbildung \ref{fig:Phi_a} dargestellt ist.
\begin{figure}[H]
  \includegraphics[height=8cm]{../../Blatt4/Plots/2_a.pdf}
  \caption{Numerisch errechnete Werte des elektrostatischen Potentials sowie analytisch abgeschätzter Monopolterm.}
  \label{fig:Phi_a}
\end{figure}
Es lässt sich erkennen, dass sich die Werte für große $x$ asymptotisch dem Monopolterm annähern.


\subsection*{Aufgabenteil b)}
Für die Ladungsverteilung
\begin{equation}
  \begin{cases}
    \rho_0\frac{x}{a}, & \lvert x \rvert < a, \lvert y \rvert < a, \lvert z \rvert < a \\
    0, & \text{sonst}
  \end{cases}
\end{equation}
wird analog zum ersten Aufgabeteil vorgegangen, also das Integral
\begin{equation}
  \Phi(x)=\int_{-1}^{1}\text{d}x'\int_{-1}^{1}\text{d}y'\int_{-1}^{1}\text{d}z'\frac{x}{\sqrt{(x-x')^2+y'^2+z'^2}}
\end{equation}
numerisch berechnet. Durch die Multipolentwicklung
ergibt sich
\begin{equation}
  \Phi(x)\approx\int_{-1}^{1}\text{d}x'\int_{-1}^{1}\text{d}y'\int_{-1}^{1}\text{d}z'\frac{x'}{\lvert x \rvert}+x'^2\frac{x}{\lvert x \rvert^3} = \frac{8x}{3\lvert x \rvert^3}
\end{equation}
wobei der Monopolterm verschwindet und nur der Dipolterm $\frac{8x}{3\lvert x \rvert^3}$ übrig bleibt. Dieser ist zusammen mit den
numerischen Ergebnissen in Abbildung \ref{fig:Phi_b} dargestellt.
\begin{figure}[H]
  \includegraphics[height=8cm]{../../Blatt4/Plots/2_b.pdf}
  \caption{Numerisch errechnete Werte des elektrostatischen Potentials sowie analytische abgeschätzter Dipolterm.}
  \label{fig:Phi_b}
\end{figure}
Auch hier lässt sich erkennen, dass sich das numerisch errechnete Potential für große $x$ dem analytisch abgeschätzten Dipolterm annähert.
