\section*{Aufgabe 0: Verständnisfragen}

\textbf{Frage 1): Wieso ist es häufig nicht sinnvoll, eine feste Schrittweite für eine Integration zu verwenden?
Lässt sich das Problem beheben?}\\

Bei zu großer Schrittweite ist der Fehler ($\sim \Delta x^2$) zu groß,
wird die Schrittweite zu klein gewählt, entstehen Rundungsfehler im Quotienten.
Der Bereich zwischen zu kleiner und zu großer Schrittweite ist meist ausreichend
groß, sodass eine numerische Berechnung möglich ist.

Desweiteren kann sich eine zu integrierende Funktion in einem Gebiet sehr schnell ändern, hier
wird für die Genauigkeit eine kleine Schrittweite benötigt, und in einem anderen
Gebiet nur schwach variieren, hier ist eine große Schrittweite sinnvoll für
eine schnellere numerische Lösung.
Dabei sind die Gebiete nicht unbedingt klar, in denen die Schrittweite
verkleinert werden kann bzw vergrößert werden muss.

Durch die relativen Fehler lässt sich abschätzen, ob in einem Gebiet die
Schrittweite erhöht werden kann oder verringert werden muss.

\noindent \,


\textbf{Frage 2): Welche anderen Probleme können bei numerischer Integration typischerweise auftreten?}\\


\begin{itemize}
  \item unendliche Integrationsgrenzen
  \item stark variierende Integranden in verschiedenen Intervallen

  \rightarrow Probleme beim Aufaddieren von sehr kleinen und sehr großen Zahlen
  \item Hauptwertintegrale
  \item Singuläre, integrable Integranden

\end{itemize}

