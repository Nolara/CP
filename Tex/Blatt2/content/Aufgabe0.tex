\section*{Aufgabe 0: Verständnisfragen}

\textbf{Frage 1): Welche physikalischen Anwendungsfälle einer Eigenwertberechnung fallen Ihnen ein?}\\

Physikalische Anwendung:

\begin{itemize}
  \item Schwingungen - Berechnungen von Eigenfrequenzen und Schwingungsmoden
  \item Basiswechsel (zBsp. Hauptachsentransformation)
  \item Quantenmechanik
      \begin{equation*}
        \hat A |v\rangle=\lambda|v\rangle
      \end{equation*}
  \item Matrixdiagonalisierung
  \begin{itemize}
    \item zBsp. $\hat A$ in der Darstellung
    \begin{equation*}
      \hat A = \sum_{i} l_i |b_i\rangle \langle b_i |
    \end{equation*}
    \item Störungstheorie
  \end{itemize}
\end{itemize}

\textbf{Frage 2): Für Aufgabe 2 und Aufgabe 3 sollen Sie einen Profiler verwenden, welcher die Zeit zwischen zwei Punkten im Programmcode stoppt. Welche anderen Arten von Profilern gibt es?}\\


Die häufigste Art von Profilern ist das Messen von Geschwindigkeiten wie in Aufgabe 2 und Aufgabe 3
Weitere Arten von Profiler sind zum Beispiel Profiler, die die Genauigkeit oder auch die Speichernutzung durch ein Programm analysieren.
Des Weiteren kann mit Hilfe von Profilern das Laufzeitverhalten von nebenläufigen Prozessen interpretiert werden.
