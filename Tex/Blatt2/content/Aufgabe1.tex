\section*{Aufgabe 1: Singulärwertzerlegung}

\subsection*{Aufgabenteil a)}
Das Bild wird mittels der Routine loadData aus der Datei service.cpp eingelesen.
Es wird eine Singuärwertzerlegung mittels Eigen::BDCSVD aus der Bibliothek Eigen durchgeführt.

\subsection*{Aufgabenteil b)}

Es wird eine Rang-$k$-Approximation gemäß
\begin{equation}
  \symbf{\tilde{A}} = \sum_{k=1}^N w_k \vec{u}_k {\vec{v}_k}^T
\end{equation}
durchgeführt.
Es werden jeweils die Bilder für $k=10,20,50$ approximiert und mit Python geplottet.
In den Abbildungen \ref{fig:k10}, \ref{fig:k20} und \ref{fig:k50} sind die jeweiligen Approximationen der Bilder dargestellt, sowie zum Vergleich das
Orginalbild in Abbildung \ref{fig:Original}.

\begin{figure}[H]
  \centering
  \fbox{\includegraphics[height=7cm]{../../Blatt2/Plots/1_10.pdf}}
  \caption{Approximation für $k=10$.}
  \label{fig:k10}
\end{figure}


\begin{figure}[H]
  \centering
  \fbox{\includegraphics[height=7cm]{../../Blatt2/Plots/1_20.pdf}}
  \caption{Approximation für $k=20$.}
  \label{fig:k20}
\end{figure}

\begin{figure}[H]
  \centering
  \fbox{\includegraphics[height=7cm]{../../Blatt2/Plots/1_50.pdf}}
  \caption{Approximation für $k=50$.}
  \label{fig:k50}
\end{figure}


\begin{figure}[H]
  \centering
  \fbox{\includegraphics[height=7cm]{../../Blatt2/Plots/1_Original.pdf}}
  \caption{Das Orginalbild.}
  \label{fig:Original}
\end{figure}

Es fällt auf, dass je höher der $k$-Wert der Approximation gewählt wird, desto mehr gleicht die Approximation des Bildes dem Orginalbild.
Das Orginalbild ist eine 512x512 Matrix mit 265 verschiedenen Grauleveln und für dessen Speicherung 304.787 Byte notwendig sind.
Die Speicherung der Matrizen für $k=10$ erfordert 161.160 Byte, was zu einer Kompressionsrate von $\SI{1.9}{\percent}$ führt.
Die Speicherung der Matrizen für $k=20$ und $k=50$ benötigen 173.456 Byte und 190.764 Byte. Diese haben eine Kompressionsrate von
$\SI{1.8}{\percent}$ und $\SI{1.6}{\percent}$.
Die Kompressionsrate $K$ wird gemäß der Formel
\begin{equation}
  K=\frac{\text{Daten}_{\text{unkomp.}}}{\text{Daten}_{\text{komp.}}}
\end{equation}
berechnet. Die angegebnen Größen der Bilder beziehen sich dabei auf die gegegebenen Größen der pdf-Dokumente.
